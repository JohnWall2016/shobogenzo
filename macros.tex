% -*- coding: utf-8 -*-

%% standalone subfiles

\let\byebye = \relax
\csname loaded\endcsname
\let\loaded = \endinput
\let\byebye = \bye

%% macros for The TeXbook

\input manmac

%% compatable with manmac.tex

\catcode `\^=7

%% font definitions

\def\song{SimSun}
\def\hei{SimHei}
\def\kai{KaiTi}
\def\fang{FangSong}
\def\yuan{YouYuan:embolden=3;extend=0.8}
%\def\yuan{FZY3JW--GB1-0:embolden=3;extend=0.95}
%
\input xeCJK-base
\xeCJKenablechecksingle
\normalspacedchars{-}
%
\def\letfont{\let\CJKfont}
\font\zhtenrm="\song" at 10pt
\let\CJKfont\zhtenrm
%
\def\setecfont#1#2#3#4{
  \expandafter\font\csname en#1\endcsname = #2 at #4pt
  \expandafter\font\csname zh#1\endcsname = "#3" at #4pt
  \expandafter\def\csname#1\endcsname{\csname en#1\endcsname\expandafter\letfont\csname zh#1\endcsname}
}

\setecfont{titlefont}{cmssdc10}{\yuan}{36}
\setecfont{chapfont}{cmssdc10}{\yuan}{36}
\setecfont{sectfont}{cmbx14}{\hei}{18}
\setecfont{ssectfont}{cmbx12}{\hei}{16}
\setecfont{sssectfont}{cmbx11}{\hei}{15}
%
\setecfont{fourteenrm}{cmr14}{\song}{14}
\setecfont{twelverm}{cmr12}{\song}{12}
\setecfont{elevenrm}{cmr11}{\song}{11}
\setecfont{tenrm}{cmr10}{\song}{10}
\setecfont{ninerm}{cmr9}{\song}{9}
\setecfont{eightrm}{cmr8}{\song}{8}
\setecfont{sevenrm}{cmr7}{\song}{7}
\setecfont{sixrm}{cmr6}{\song}{6}
\setecfont{fiverm}{cmr5}{\song}{5}
%
\setecfont{fourteenbf}{cmbx14}{\hei}{14}
\setecfont{twelvebf}{cmbx12}{\hei}{12}
\setecfont{elevenbf}{cmbx11}{\hei}{11}
\setecfont{tenbf}{cmbx10}{\hei}{10}
\setecfont{ninebf}{cmbx9}{\hei}{9}
\setecfont{eightbf}{cmbx8}{\hei}{8}
\setecfont{sevenbf}{cmbx7}{\hei}{7}
\setecfont{sixbf}{cmbx6}{\hei}{6}
\setecfont{fivebf}{cmbx5}{\hei}{5}
%
\setecfont{fourteentt}{cmtt14}{\fang}{14}
\setecfont{twelvett}{cmtt12}{\fang}{12}
\setecfont{eleventt}{cmtt11}{\fang}{11}
\setecfont{tentt}{cmtt10}{\fang}{10}
\setecfont{ninett}{cmtt9}{\fang}{9}
\setecfont{eighttt}{cmtt8}{\fang}{8}
\setecfont{seventt}{cmtt7}{\fang}{7}
\setecfont{sixtt}{cmtt6}{\fang}{6}
\setecfont{fivett}{cmtt5}{\fang}{5}
%
\setecfont{fourteensl}{cmsl14}{\kai}{14}
\setecfont{twelvesl}{cmsl12}{\kai}{12}
\setecfont{elevensl}{cmsl11}{\kai}{11}
\setecfont{tensl}{cmsl10}{\kai}{10}
\setecfont{ninesl}{cmsl9}{\kai}{9}
\setecfont{eightsl}{cmsl8}{\kai}{8}
\setecfont{sevensl}{cmsl7}{\kai}{7}
\setecfont{sixsl}{cmsl6}{\kai}{6}
\setecfont{fivesl}{cmsl5}{\kai}{5}
%
\setecfont{fourteenit}{cmti14}{\kai}{14}
\setecfont{twelveit}{cmti12}{\kai}{12}
\setecfont{elevenit}{cmti11}{\kai}{11}
\setecfont{tenit}{cmti10}{\kai}{10}
\setecfont{nineit}{cmti9}{\kai}{9}
\setecfont{eightit}{cmti8}{\kai}{8}
\setecfont{sevenit}{cmti7}{\kai}{7}
\setecfont{sixit}{cmti6}{\kai}{6}
\setecfont{fiveit}{cmti5}{\kai}{5}

\def\ST#1{\font\tempfont="\song" at #1pt\letfont\tempfont}
\def\HT#1{\font\tempfont="\hei" at #1pt\letfont\tempfont}
\def\KT#1{\font\tempfont="\kai" at #1pt\letfont\tempfont}
\def\FS#1{\font\tempfont="\fang" at #1pt\letfont\tempfont}
\def\YY#1{\font\tempfont="\yuan" at #1pt\letfont\tempfont}

%% some redefinition of manmac.tex

\font\titlefonta=cmssdc10 at 20pt

\font\fourteeni=cmmi14
\font\fourteensy=cmsy14
\font\fourteenex=cmex14
\def\fourteenpoint{\def\rm{\fam0\fourteenrm}%
  \textfont0=\fourteenrm \scriptfont0=\elevenrm \scriptscriptfont0=\ninerm
  \textfont1=\fourteeni \scriptfont1=\eleveni \scriptscriptfont1=\ninei
  \textfont2=\fourteensy \scriptfont2=\elevensy \scriptscriptfont2=\ninesy
  \textfont3=\fourteenex \scriptfont3=\fourteenex \scriptscriptfont3=\fourteenex
  \def\it{\fam\itfam\fourteenit}%
  \textfont\itfam=\fourteenit
  \def\sl{\fam\slfam\fourteensl}%
  \textfont\slfam=\fourteensl
  \def\bf{\fam\bffam\fourteenbf}%
  \textfont\bffam=\fourteenbf \scriptfont\bffam=\elevenbf
   \scriptscriptfont\bffam=\ninebf
  \def\tt{\fam\ttfam\fourteentt}%
  \textfont\ttfam=\fourteentt
  \tt \ttglue=.5em plus.25em minus.15em
  \normalbaselineskip=23pt
  \def\MF{{\manual META}\-{\manual FONT}}%
  \let\sc=\elevenrm
  \let\big=\fourteenbig
  \setbox\strutbox=\hbox{\vrule height10.5pt depth5.5pt width\z@}%
  \normalbaselines\rm}
%
\font\twelvei=cmmi12
\font\twelvesy=cmsy12
\font\twelveex=cmex12
\def\twelvepoint{\def\rm{\fam0\twelverm}%
  \textfont0=\twelverm \scriptfont0=\ninerm \scriptscriptfont0=\sevenrm
  \textfont1=\twelvei \scriptfont1=\ninei \scriptscriptfont1=\seveni
  \textfont2=\twelvesy \scriptfont2=\ninesy \scriptscriptfont2=\sevensy
  \textfont3=\twelveex \scriptfont3=\twelveex \scriptscriptfont3=\twelveex
  \def\it{\fam\itfam\twelveit}%
  \textfont\itfam=\twelveit
  \def\sl{\fam\slfam\twelvesl}%
  \textfont\slfam=\twelvesl
  \def\bf{\fam\bffam\twelvebf}%
  \textfont\bffam=\twelvebf \scriptfont\bffam=\ninebf
   \scriptscriptfont\bffam=\sevenbf
  \def\tt{\fam\ttfam\twelvett}%
  \textfont\ttfam=\twelvett
  \tt \ttglue=.5em plus.25em minus.15em
  \normalbaselineskip=19pt
  \def\MF{{\manual META}\-{\manual FONT}}%
  \let\sc=\ninerm
  \let\big=\twelvebig
  \setbox\strutbox=\hbox{\vrule height9.5pt depth4.5pt width\z@}%
  \normalbaselines\rm}
%
\font\eleveni=cmmi11
\font\elevensy=cmsy11
\font\elevenex=cmex11
\def\elevenpoint{\def\rm{\fam0\elevenrm}%
  \textfont0=\elevenrm \scriptfont0=\eightrm \scriptscriptfont0=\sixrm
  \textfont1=\eleveni \scriptfont1=\eighti \scriptscriptfont1=\sixi
  \textfont2=\elevensy \scriptfont2=\eightsy \scriptscriptfont2=\sixsy
  \textfont3=\elevenex \scriptfont3=\elevenex \scriptscriptfont3=\elevenex
  \def\it{\fam\itfam\elevenit}%
  \textfont\itfam=\elevenit
  \def\sl{\fam\slfam\elevensl}%
  \textfont\slfam=\elevensl
  \def\bf{\fam\bffam\elevenbf}%
  \textfont\bffam=\elevenbf \scriptfont\bffam=\eightbf
   \scriptscriptfont\bffam=\sixbf
  \def\tt{\fam\ttfam\eleventt}%
  \textfont\ttfam=\eleventt
  \tt \ttglue=.5em plus.25em minus.15em
  \normalbaselineskip=16pt
  \def\MF{{\manual META}\-{\manual FONT}}%
  \let\sc=\eightrm
  \let\big=\elevenbig
  \setbox\strutbox=\hbox{\vrule height9pt depth4pt width\z@}%
  \normalbaselines\rm}
%
\def\tenpoint{\def\rm{\fam0\tenrm}%
  \textfont0=\tenrm \scriptfont0=\sevenrm \scriptscriptfont0=\fiverm
  \textfont1=\teni \scriptfont1=\seveni \scriptscriptfont1=\fivei
  \textfont2=\tensy \scriptfont2=\sevensy \scriptscriptfont2=\fivesy
  \textfont3=\tenex \scriptfont3=\tenex \scriptscriptfont3=\tenex
  \def\it{\fam\itfam\tenit}%
  \textfont\itfam=\tenit
  \def\sl{\fam\slfam\tensl}%
  \textfont\slfam=\tensl
  \def\bf{\fam\bffam\tenbf}%
  \textfont\bffam=\tenbf \scriptfont\bffam=\sevenbf
   \scriptscriptfont\bffam=\fivebf
  \def\tt{\fam\ttfam\tentt}%
  \textfont\ttfam=\tentt
  \tt \ttglue=.5em plus.25em minus.15em
  \normalbaselineskip=15pt
  \def\MF{{\manual META}\-{\manual FONT}}%
  \let\sc=\eightrm
  \let\big=\tenbig
  \setbox\strutbox=\hbox{\vrule height8.5pt depth3.5pt width\z@}%
  \normalbaselines\rm}
%
\def\ninepoint{\def\rm{\fam0\ninerm}%
  \textfont0=\ninerm \scriptfont0=\sixrm \scriptscriptfont0=\fiverm
  \textfont1=\ninei \scriptfont1=\sixi \scriptscriptfont1=\fivei
  \textfont2=\ninesy \scriptfont2=\sixsy \scriptscriptfont2=\fivesy
  \textfont3=\tenex \scriptfont3=\tenex \scriptscriptfont3=\tenex
  \def\it{\fam\itfam\nineit}%
  \textfont\itfam=\nineit
  \def\sl{\fam\slfam\ninesl}%
  \textfont\slfam=\ninesl
  \def\bf{\fam\bffam\ninebf}%
  \textfont\bffam=\ninebf \scriptfont\bffam=\sixbf
   \scriptscriptfont\bffam=\fivebf
  \def\tt{\fam\ttfam\ninett}%
  \textfont\ttfam=\ninett
  \tt \ttglue=.5em plus.25em minus.15em
  \normalbaselineskip=14pt
  \def\MF{{\manual hijk}\-{\manual lmnj}}%
  \let\sc=\sevenrm
  \let\big=\ninebig
  \setbox\strutbox=\hbox{\vrule height8pt depth3pt width\z@}%
  \normalbaselines\rm
  \ST{9}}
%
\def\eightpoint{\def\rm{\fam0\eightrm}%
  \textfont0=\eightrm \scriptfont0=\sixrm \scriptscriptfont0=\fiverm
  \textfont1=\eighti \scriptfont1=\sixi \scriptscriptfont1=\fivei
  \textfont2=\eightsy \scriptfont2=\sixsy \scriptscriptfont2=\fivesy
  \textfont3=\tenex \scriptfont3=\tenex \scriptscriptfont3=\tenex
  \def\it{\fam\itfam\eightit}%
  \textfont\itfam=\eightit
  \def\sl{\fam\slfam\eightsl}%
  \textfont\slfam=\eightsl
  \def\bf{\fam\bffam\eightbf}%
  \textfont\bffam=\eightbf \scriptfont\bffam=\sixbf
   \scriptscriptfont\bffam=\fivebf
  \def\tt{\fam\ttfam\eighttt}%
  \textfont\ttfam=\eighttt
  \tt \ttglue=.5em plus.25em minus.15em
  \normalbaselineskip=12pt
  \def\MF{{\manual opqr}\-{\manual stuq}}%
  \let\sc=\sixrm
  \let\big=\eightbig
  \setbox\strutbox=\hbox{\vrule height7pt depth2pt width\z@}%
  \normalbaselines\rm}

\def\leftheadline{\hbox to \pagewidth{%
    \vbox to 10pt{}% strut to position the baseline
    {\fourteenbf\folio}\hskip10pt\fourteenit\rbook\hfil% folio to left of text
    %\rhead% running head flush left
    }}
\def\rightheadline{\hbox to \pagewidth{%
    \vbox to 10pt{}% strut to position the baseline
    %\rhead% running head flush right
    \hfil\fourteenit\rhead\hskip10pt {\fourteenbf\folio}% folio to right of text
    }}

\newcount\footnoteno
\let\oldfootnote=\footnote  
\def\footnotemy#1{\oldfootnote{}{{\hang\smash{\hbox to 0pt{%
          \hskip-\hangindent\hskip+0.5pc\the\footnoteno\hfill}}}#1}%
  \global\advance\footnoteno by 1}
\def\footnote#1{$\!\!^{\the\footnoteno}${\parindent=1.7pc\footnotemy{#1}}}
\def\myfootnote#1{$\!\!^{\the\footnoteno}${\parindent=1.7pc\footnotemy{【译注】#1}}}
\def\myfootnoteaq#1{$\!\!^{\the\footnoteno}${\parindent=1.7pc\footnotemy{【译注】#1}}}

\def\folio{\ifnum\pageno<0 \uppercase\expandafter{\romannumeral-\pageno} \else\number\pageno \fi}

\def\beginchapter#1 \par{\global\exno=0
  \subsecno=0
  \footnoteno=1
  \titlepage
  \bookmark{1}{#1}
  \def\MF{{\manual 89:;<=>:}} % slant the logo
  \xdef\rhead{#1\unskip}
  {\def\TeX{T\kern-.2em\lower.5ex\hbox{E}\kern-.06em X}
    \def\MF{{\vbox to30pt{}\manual ()*+,-.*}}
    \centerline{\titlefont#1}
    \vskip 4pc
  } % output the chapter title page
  \fourteenpoint}

\def\beginchaptern#1#2 \par{\global\exno=0
  \subsecno=0
  \footnoteno=1
  \titlepage
  \bookmark{1}{#1}
  \def\MF{{\manual 89:;<=>:}} % slant the logo
  \xdef\rhead{#1\unskip}
  {\def\TeX{T\kern-.2em\lower.5ex\hbox{E}\kern-.06em X}
    \def\MF{{\vbox to30pt{}\manual ()*+,-.*}}
    \centerline{\titlefont#1}
    \vskip 1pc
    \centerline{\fourteenit#2}
    \vskip 2pc
  } % output the chapter title page
  \fourteenpoint}

\outer\def\endchapter{\vfill\eject%
  \ifodd\pageno\else\null\vfill\eject\fi}

\outer\def\subsection#1. {\medbreak\advance\subsecno by 1
  \bookmark{2}{\the\subsecno. #1}
  \noindent{\bf \the\subsecno.\enspace#1.\enspace}}

\outer\def\subsectnon#1 \par{\bigbreak\medbreak
  \def\rhead{\def\1{}#1\unskip}
  \bookmark{2}{#1}
  \noindent{\sectfont #1\enspace}\medbreak}

\outer\def\subsectnontp#1 #2 \par{\bigbreak\medbreak
  \def\rhead{\def\1{}#1\unskip}
  \bookmark{2}{#1}
  \noindent{\sectfont #1\enspace}\par%
  \noindent{\KT{18}#2\enspace}\medbreak}

\outer\def\ssubsectnon#1 \par{\medbreak
  \def\sshid##1{}
  \bookmark{3}{#1}
  \def\sshid##1{##1}
  \noindent{\ssectfont #1\enspace}\smallbreak}

\outer\def\ssubsectnontp#1 #2 \par{\medbreak
  \def\sshid##1{}
  \bookmark{3}{#1}
  \def\sshid##1{##1}
  \noindent{\ssectfont #1\enspace}\par%
  \noindent{\KT{16}#2\enspace}\smallbreak}

\outer\def\ssubsectnonthp#1 #2 #3\par{\medbreak
  \def\sshid##1{}
  \bookmark{3}{#1}
  \def\sshid##1{##1}
  \noindent{\ssectfont #1\enspace}\par%
  \noindent{\KT{16}#2\enspace}\par%
  \noindent{\KT{16}#3\enspace}\smallbreak}

\outer\def\sssubsectnon#1 \par{\medbreak
  \bookmark{4}{#1}
  {\sssectfont #1\enspace}\smallbreak}

\outer\def\sssubsectnoni#1 \par{\medbreak
  \bookmark{4}{#1}
  \noindent{\sssectfont #1\enspace}\smallbreak}

\outer\def\sssubsectnonb#1 \par{\medbreak
  {\sssectfont #1\enspace}\smallbreak}

\outer\def\sssubsectnonib#1 \par{\medbreak
  \noindent{\sssectfont #1\enspace}\smallbreak}

\outer\def\ssssubsectnonb#1 \par{\medbreak
  \bookmark{5}{#1}
  {\sssectfont #1\enspace}\smallbreak}

\outer\def\ssssubsectnonib#1 \par{\medbreak
  \def\sshid##1{}
  \bookmark{5}{#1}
  \def\sshid##1{##1}
  \noindent{\sssectfont #1\enspace}\smallbreak}

\def\beginlines{\par\begingroup\baselineskip=12pt\nobreak\medskip\parindent\z@ \obeylines
  \hrule\kern1pt\nobreak \everypar{\strut}}

\def\pdfdestx#1{%
  \special{pdf:dest (#1) [@thispage /FitH @ypos]}%
}
\def\pdflinkx#1#2{%
  \special{pdf:bann << /Type/Annot/Subtype/Link  /Border [0 0 0] /A << /S/GoTo/D (#1) >> >>}%
  \special{pdf:bc [0 0 1]}#2\special{pdf:ec}%
  \special{pdf:eann}%
}

\def\note#1{ {\nineit #1}}

\def\word#1#2{{\underbar{#1}}$\!${\nineit #2}}

\def\uline#1{$\!\!$\underbar{#1}}

\proofmodefalse % this should be false when making camera-ready copy

\def\frac#1/#2{\leavevmode\kern.1em
  \raise.5ex\hbox{\the\scriptfont0 #1}\kern-.1em
  /\kern-.15em\lower.25ex\hbox{\the\scriptfont0 #2}}

%% from old bookall.tex
\baselineskip=18pt
\lineskip=2pt
\twelvepoint

\parindent=2.4pc

%% paper dimensions

\hsize=34pc
\vsize=44pc
\pagewidth=\hsize
\pageheight=\vsize
\pdfpagewidth=7.7in \pdfpageheight=10in

%% pdf bookmarks

\def\gbmark#1{\special{pdf: out 1 << /Title <FEFF#1> /Dest [ @thispage /FitH @ypos ] >>}}

\special{pdf: docview << /PageMode /UseOutlines >>}
\def\bookmark#1#2{%
  \begingroup%
  \def\1{}\def\linebreak{}\def\TeX{TeX}\def\noindent{}\def\\{}\def\hfil{}\def\quad{}%
  \special{pdf:out #1 << /Title (#2) /Dest [ @thispage /FitH @ypos ] >>}%
  \endgroup%
}

%% add original page numbers

\newcount\origpageno

\def\strutdepth{\dp\strutbox}
\def\1{\strut\vadjust{\kern-\strutdepth\pagenotext}%
  \pdfdestx{page\the\origpageno}\global\advance\origpageno by 1}
\def\pagenotext{\vtop to \strutdepth{%
  \baselineskip\strutdepth\vss\llap{\tt[\the\origpageno]\quad}\null}}

%% compatable with manmac.tex

\catcode `\^=13
