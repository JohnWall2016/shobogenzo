\input macros

\beginchaptern{现成公案}{}

诸法之为佛法时节,即有迷有悟,有修行,有生死,有诸佛,有众生。万法非属我之时节,无迷无悟,无诸佛,无众生,无生无死。

佛道原本跳出丰俭,故有生灭,有迷悟,有众生与佛。虽言如是,花于爱惜而凋落,草于弃嫌而丛生。

强运自己修证万法为迷,万法进前修证自己为悟。大悟迷者为诸佛,大迷悟者为众生。更有悟上得悟之汉,迷中又迷之汉。

诸佛正为诸佛之时,毋需知觉自己之为诸佛。然则,其为证佛者,证佛而不休止。举身心见取(形)色,举身心听取音声,虽为会取,然非如镜中映影,非如水与月之喻。证一方时,另一方则暗。

所谓学佛道者,即学自己也。学自己者,即忘自己也。忘自己者,为万法所证也。为万法所证者,即令自己之身心及他人之身心脱落也。若有悟迹休歇,即令休歇之悟迹长长流出。

人始求法时,则远离法之边际;法既正传于自己时,即为本分人也。人乘舟而行,举目视岸,误以为岸移,俯视舟时,即知舟行。如是之理,乱想身心而辨肯万法,则误执自心自性为常住。如能亲归行李于个里,则可了知万法不属我之道理。

薪燃成灰,不复更成薪。虽然如是,不得见取灰后薪前。当知薪住薪之法位,有先有后;虽有前后,前后际断。灰住灰之法位,有后有前。如彼之薪燃成灰后不复再成薪故,人之死后,不复回生。然则,不言生而死者,佛法之定说也,故言不生。死而不复回生者,法轮之所定之佛转也,故云不灭。生乃一时之法位也,死亦乃一时之法位也,比如冬春。不思冬后而春,不言春后而夏。

\endchapter
