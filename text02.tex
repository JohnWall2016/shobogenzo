\input macros

\beginchaptern{现成公案}{}

诸法之为佛法时节,即有迷有悟,有修行,有生死,有诸佛,有众生。万法非属我之时节,无迷无悟,无诸佛,无众生,无生无死。

佛道原本跳出丰俭,故有生灭,有迷悟,有众生与佛。虽言如是,花于爱惜而凋落,草于弃嫌而丛生。

强运自己修证万法为迷,万法进前修证自己为悟。大悟迷者为诸佛,大迷悟者为众生。更有悟上得悟之汉,迷中又迷之汉。

诸佛正为诸佛之时,毋需知觉自己之为诸佛。然则,其为证佛者,证佛而不休止。举身心见取(形)色,举身心听取音声,虽为会取,然非如镜中映影,非如水与月之喻。证一方时,另一方则暗。

所谓学佛道者,即学自己也。学自己者,即忘自己也。忘自己者,为万法所证也。为万法所证者,即令自己之身心及他人之身心脱落也。若有悟迹休歇,即令休歇之悟迹长长流出。

人始求法时,则远离法之边际;法既正传于自己时,即为本分人也。人乘舟而行,举目视岸,误以为岸移,俯视舟时,即知舟行。如是之理,乱想身心而辨肯万法,则误执自心自性为常住。如能亲归行李于个里,则可了知万法不属我之道理。

薪燃成灰,不复更成薪。虽然如是,不得见取灰后薪前。当知薪住薪之法位,有先有后;虽有前后,前后际断。灰住灰之法位,有后有前。如彼之薪燃成灰后不复再成薪故,人之死后,不复回生。然则,不言生而死者,佛法之定说也,故言不生。死而不复回生者,法轮之所定之佛转也,故云不灭。生乃一时之法位也,死亦乃一时之法位也,比如冬春。不思冬后而春,不言春后而夏。

人之得悟,如月映水,月不湿,水不破。光虽广大,映于寸尺之水,全月弥天,既映草露,亦宿一滴之水。悟不破人者,如月不穿水。人不碍悟者,如滴露不碍天月也。深者,高之份量也。时节之长短,当检点大水小水,辨取天月之广狭也。

法尚未参饱于身心,以为法已满足;法若充足身心,尚觉法之不足。如人乘船出无山之海,眺望四方,则见海唯圆形,更不见不同之相。然则大海既非圆形,亦非方形,以其海德,无有穷尽,(见)如宫殿,(见)如璎珞。唯眼所及,而见海为圆形。如彼故,万法亦然,尘中格外,虽备诸多样相,人皆只见取会取参学眼力之所及也。若欲知晓万法之家风,当知方形圆形之外,更有海德山德,无穷无尽,眼外更有世界。非但身旁如此,当知当下一念、一滴之水,亦皆如是。

鱼游水中,水无际涯;鸟飞天空,天无界限。然自昔以来,鱼不离水,鸟不离天。唯用大之时则使大,用小之时,则使小而已。如是头头无有不尽之边际,处处无不踏翻,然鸟若出天,当即死之;鱼若出水,亦当即死之。当知(鱼)以水为命,(鸟)以天为命。有以鸟为命者,有以鱼为命者。以命为鸟,以命为鱼。此外、更可进步(言之)。有修证,有寿者命者,(其理)亦皆如是。然则,倘若鱼穷究水后而游水,鸟穷究天后而飞天,则于水于天皆不得道,不得处。若得此所,则此行李随之而现成公案;若得是道,则此行李随之而现成公案。此道此所,非大非小,非自非他,非先有,亦非今现,故有如此(之理)也。

然则人若修证佛道,即得一法通一法也,遇一行修一行也。以之有处所,道通达,而所知之境之不明者,盖此知与佛法之究竟同生同参故也。切勿以得处为自己之知见,以虑知为知之。证究虽即现成,然密有未必现成;现成不必(如此)也。

麻浴山宝彻禅师用扇子,时僧来问:“风性常住,无处不周,和尚以何更用扇子。”师曰:“汝只知风性常住之理,却不知无处不周底道理。”僧曰:“如何是无处不周底道理?”时师只管使扇。僧礼拜。

佛法之证验,正传之活路,其如是也。以是常住者而言无须使扇,不用时而言有风吹者,实不知常住,亦不知风性也。风性以其常住故,佛家之风现成大地之黄金,参熟长河之酥酪。

\vskip 2pc
正法眼藏现成公案第一

是为天福元年中秋顷书与镇西之俗弟子杨光秀。

建长壬子拾勒

\endchapter
