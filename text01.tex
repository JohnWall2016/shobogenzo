% -*- coding: utf-8 -*-

\input macros

\beginchaptern{办道话}{修道篇}

\origpageno=1
\pageno=1

诸佛如来,皆单传妙法,证阿耨菩提\note{阿耨多罗三藐三菩提,无上正等正觉},有最上无为之妙术,是唯佛授之于佛而无旁邪\note{误入歧途},即\note{而}自受用三昧\note{自己受益的禅定},是其标准也。

游化此三昧,当以端坐参禅为正门。是法虽人人分上丰备,然未修者不现,未证者不得。放则满手,岂一多之际乎?语则溢口,纵横无限。

诸佛常住持此中,各个方面不遗知觉;群生常受用此中,各个知觉方面不露。

今授之功夫办道者,乃于证上现万法,于出路行一如也。其超关脱落时,岂拘此节目乎?

予自发心求法以来,于我朝诸方,寻访知识,因见建仁全公,相随霜华忽历九载,聊闻临济家风。全公者,祖师西和尚之上足,独正传无上之佛法,余辈不敢并比。

予又赴大宋国,访知识于两浙,闻家风于五门。遂参见太白峰净禅师,一生参学大事于兹了毕。尔后,大宋绍定初,返回本乡,即以弘法救生为怀,尚如重坦荷肩。

然放下弘通(佛法)之心,以待激扬之时故。且云游萍寄,将期闻先哲之风范。但有自不拘名利,以道念为先,真实参学者乎?徒被邪师所惑,猥覆正解,空枉自狂,久沉迷乡,以何而长般若正种,得得道之时哉?贫道今既以云游萍寄为事,将参访何山川?以怜是故,遂将于大宋国所闻之禅林风规,禀持之知识玄旨,记之集之,留与参学闲道之人,令知晓佛家之正法,此乃真诀也。

曰:大师释尊,灵山会上付法迦叶,祖祖正传而至菩提达摩尊者。尊者亲赴神丹国,付法慧可大师,是东地佛法传来之始也。

如是单传,亲至六祖大鉴禅师。是时,真实之佛法,方流演东汉,不拘节目之旨,于焉而显。于时六祖有二神足,为南岳怀让、青原行思。皆传持佛印,同为人天导师。其二派流通,善开五门,谓法眼宗、曹洞宗、云门宗、临济宗也。现在大宋国,独临济宗遍于天下。五宗虽异,唯一佛心印也。

大宋国自后汉以来,虽教籍垂迹,广播天下,然雌雄未定。祖师西来之后,直截葛藤根源,纯一佛法,于焉弘通。亦愿我国有如是之事也。