% -*- coding: utf-8 -*-

\input macros

\beginchaptern{办道话}{修道篇}

\origpageno=1
\pageno=1

诸佛如来,皆单传妙法,证\word{阿耨菩提}{阿耨多罗三藐三菩提,无上正等正觉},有最上无为之妙术,是唯佛授之于佛而无\word{旁邪}{误入歧途},\word{即}{而}自\word{受用三昧}{自己受益的禅定},是其标准也。

游化\note{云游教化,传授}此三昧,当以\word{端}{肩平背直}坐参禅为正门。是法虽\word{人人分上丰备}{佛性人人生而具备},然未修者不\word{现}{显现出},未证者不\word{得}{体验到}。\word{放则满手}{不握着它反而充满双手},\word{岂一多之际乎}{无数量的限制}?\word{语则溢口}{言语无法尽述},纵横无限。

诸佛常\word{住持}{住留}此中,\word{各个方面不遗知觉}{不留痕迹};群生常\word{受用}{受益}此中,\word{各个知觉方面不露}{而不自知}。

今授之\word{功夫}{身体力行}\word{办道者}{修道方法,此处指坐禅},乃\word{于}{在}\word{证}{证悟}上现万法,于\word{出路}{解脱之道上}行\word{一如}{真如,不异之真理}也。其\word{超关脱落}{超越烦恼,解脱生死}时,岂拘此\word{节目}{木干交接处为节,纹理纠结处为目,此处指束缚人的概念理论}乎?

予自发心求法以来,于我朝诸方,寻访知识,因见\word{建仁}{京都建仁寺}\word{全公}{佛树房明全,荣西的弟子,宋时与道元一同入宋求法,后客死在中国},相随霜华忽历九载,\word{聊}{略}闻临济家风。全公者,祖师\word{西和尚}{明庵荣西,曾入宋求法,将临济宗传入日本,被奉为日本临济宗创始人}之上足,独正传无上之佛法,余辈不敢并比。

予又赴大宋国,访知识于\word{两浙}{浙江},闻家风于\word{五门}{禅宗五宗:曹洞宗、临济宗、沩仰宗、法眼宗、云门宗}。遂参见\word{太白峰}{浙江宁波天童山}\word{净}{如净}禅师,一生参学大事于兹了毕。尔后,大宋\word{绍定}{宋年号}初,返回本乡,即以弘法救生为怀,\word{尚如}{犹如}重\word{坦}{担}荷肩。

然放下\word{弘通}{弘扬}(佛法)之心,以待\word{激扬之时}{风云际会之时}故。且\word{云游萍寄}{四处游历},将\word{期}{希望}\word{闻}{传播}先哲之风范。但有\word{自}{自身}不拘名利,以道念为先,真实参学者乎?徒被邪师所惑,\word{猥覆}{歪曲遮挡}正解,空枉自狂,久沉迷乡,以何而长\word{般若}{智慧}正种,得得道之时哉?贫道今既以云游萍寄为事,将参访何山川\note{我在四处游历,求法者如何能寻得正法}?以怜是故,遂将于大宋国所闻之禅林\word{风规}{风范规仪},禀持之知识玄旨,记之集之,留与参学\word{闲道}{非正道}之人,令知晓佛家之正法,此乃真诀也。

曰:大师\word{释尊}{释迦牟尼佛},灵山会上付法\word{迦叶}{摩诃迦叶},祖祖正传而至\word{菩提达摩}{初祖}尊者。尊者亲赴\word{神丹国}{中国},付法\word{慧可}{二祖}大师,是东地佛法传来之始也。

如是\word{单传}{一祖传一祖},亲至六祖\word{大鉴禅师}{慧能或惠能}。是时,真实之佛法,方流演东汉,不拘节目之旨,于焉而\word{显}{形成}。于时六祖有二神足,为\uline{南岳怀让}、\uline{青原行思}。皆传持\word{佛印}{佛心如印契永久不变},同为人天导师。其二派流通,善开五门,谓法眼宗、曹洞宗、云门宗、临济宗\note{还有沩仰宗}也。现在大宋国,独临济宗遍于天下。五宗虽异,唯一佛心印也。

大宋国自\word{后汉}{东汉}以来,虽\word{教籍}{佛教典籍}\word{垂迹}{佛、菩萨化身说法事迹},广播天下,然雌雄未定。祖师西来之后,直截葛藤根源,纯一佛法,于焉弘通。亦愿我国有如是之事也。

谓住持佛法之诸祖并诸佛,皆以端坐、依行自受用三昧为其开悟之正道。\word{西天}{印度}\word{东地}{中国},得悟之人,概依其\word{风}{风气习惯}。此乃因\word{师资}{师徒}秘密正传妙术,禀持真实秘诀之故也。

\word{宗门}{禅宗门庭}\word{正传}{正宗传授}云:此单传正直之佛法,最上中之最上也。自参见\word{知识}{禅宗大师}始,勿须更烧香、礼拜、念佛、修忏、看经;只管打坐,得\word{身心脱落}{超脱烦恼}。

人\word{虽}{即使}\word{一时}{一时半刻}于\word{三业}{身、口、意三种活动}\word{标}{示现}佛心印,端坐于三昧时,则遍法界皆为佛印,尽虚空悉成证悟,是故,于诸佛如来,增本地之法乐,新觉道之庄严。及\word{十方}{东、西、南、北、东南、西南、东北、西北、上、下十个方位}\word{法界}{各种事物的现象及其本质}、\word{三途}{地狱、饿鬼、畜生}\word{六道}{“三途”加修罗、人、天}之群类,皆一时身心明净,证大解脱\word{地}{境界},现本来面目,诸法皆会证正觉,万物共使用佛身,倏然一超证会之边际,端坐\word{觉树王}{菩提树},一时转无等等大法轮\note{传法},开演\word{究竟}{最终、终极}之深般若。

此等\word{等正觉}{正确的觉悟},更回光返照\note{返回你身上},\word{冥资通合}{暗中资助}故,此坐禅人,便\word{霍尔}{疾速}身心脱落,截断\word{从来}{之前}\word{杂秽}{杂乱不纯}之\word{知见}{认知见解}\word{思量}{思虑考量},\word{会}{领会}\word{证}{证悟}\word{天}{天然}\word{真}{本真}佛法,\word{悉于}{在全部}\word{遍}{遍布}\word{微尘际}{极细小的地方}之诸佛如来道场,\word{助发}{帮忙发起}\word{佛事}{礼佛修道之事},\word{广被}{广泛提供}佛之向上机\note{礼佛的机会},\word{激扬}{推崇宣扬}佛之向上法\note{学佛的法门},是时,十方法界之土地、草木、墙壁、瓦砾,亦皆作佛事;蒙其所兴之风水利益\note{因以上佛事而受益}者,皆被甚妙不可思议之\word{佛化}{佛之化身}所冥资,\word{显其亲证}{展示佛之所亲自证悟的妙谛};受用此水火之群类者,皆\word{周旋本证之佛化}{流转于本初证悟之佛的化身中}。所以,与此等群类共住同语者,亦悉皆互\word{备}{得到}无穷之\word{佛德}{佛之功德},\word{辗转}{不断}\word{广作}{广泛修行},使无尽、无间断、不可思议、不可称量之佛法流通于遍法界之内外。然则,令诸等(坐禅)\word{当}{中}人不被知觉所\word{昏}{迷惑}者,\word{以}{因}\word{于}{在}静中无\word{造作}{执著攀缘},\word{直证}{直接证悟}故也。若如\word{凡流}{普通人}\word{想}{所认为的},以\word{修}{坐禅}\word{证}{证悟}为\word{两段}{两码事},\word{则各互当觉知}{则可以分别认知}。\word{若混入觉知}{如果可以分别认知},即非\word{证则}{证悟的本质},\word{以于证则迷情不及故}{因为证悟的本质非分别之心可以企及的}。

又于心境虽有静中证入、悟出,而以自受用之境界故,不动一尘,不坏一相,广大之佛事,作甚深微妙之佛化。此化道所及之草木土地,皆放大光明,说深妙法,无有穷尽。草木墙壁能为凡圣含灵宣扬,凡圣含灵返为草木墙壁演畅。自觉觉他之境界,自本以来,具证相,无有欠缺;证则行之,而无懈怠之时。

由是,虽一人一时坐禅,以其与诸法相冥资,与诸时互圆通故,于无尽法界中,于过去、未来、现在,作常恒之佛化道事。彼彼共为一等之同修、同证也,非唯坐上之修,如击空之响,撞之前后,妙声绵绵。何限如此乎?百头(万事)皆于本来面目具本来修行,非可臆测。

须知即便十方无量恒河沙数诸佛,皆共励力,以佛之智慧计知一人坐禅之功德,亦不得其边际也。

今闻此坐禅之功德高大,(然)愚人疑谓:佛法有多门,何故偏劝坐禅耶?

示曰:是以佛法之正门故。

