% -*- coding: utf-8 -*-

\input macros

\beginchaptern{办道话}{修道篇}

\origpageno=1
\pageno=1

诸佛如来,皆单传妙法,证阿耨菩提\note{阿耨多罗三藐三菩提,无上正等正觉},有最上无为之妙术,是唯佛授之于佛而无旁邪\note{误入歧途},即\note{而}自受用三昧\note{自己受益的禅定},是其标准也。

游化\note{云游教化,传授}此三昧,当以端\note{肩平背直}坐参禅为正门。是法虽人人分上丰备\note{佛性人人生而具备},然未修者不现\note{显现出},未证者不得\note{体验到}。放则满手\note{不握着它反而充满双手},岂一多之际\note{无数量的限制}乎?语则溢口\note{言语无法尽述},纵横无限。

诸佛常住持\note{住留}此中,各个方面不遗知觉\note{不留痕迹};群生常受用\note{受益}此中,各个知觉方面不露\note{而不自知}。

今授之功夫\note{身体力行}办道者\note{修道方法,此处指坐禅},乃于\note{在}证\note{证悟}上现万法,于出路\note{解脱之道上}行一如\note{真如,不异之真理}也。其超关脱落\note{超越烦恼,解脱生死}时,岂拘此节目\note{木干交接处为节,纹理纠结处为目,此处指束缚人的概念理论}乎?

予自发心求法以来,于我朝诸方,寻访知识,因见建仁\note{京都建仁寺}全公\note{佛树房明全,荣西的弟子,宋时与道元一同入宋求法,后客死在中国},相随霜华忽历九载,聊\note{略}闻临济家风。全公者,祖师西和尚\note{明庵荣西,曾入宋求法,将临济宗传入日本,被奉为日本临济宗创始人}之上足,独正传无上之佛法,余辈不敢并比。

予又赴大宋国,访知识于两浙\note{浙江},闻家风于五门\note{禅宗五宗:曹洞宗、临济宗、沩仰宗、法眼宗、云门宗}。遂参见太白峰\note{浙江宁波天童山}净禅师\note{如净},一生参学大事于兹了毕。尔后,大宋绍定\note{宋年号}初,返回本乡,即以弘法救生为怀,尚如\note{犹如}重坦\note{担}荷肩。

然放下弘通\note{弘扬}(佛法)之心,以待激扬之时\note{风云际会之时}故。且云游萍寄\note{我四处游历},将期闻先哲之风范\note{传播先哲风范}。但有自\note{来自}不拘名利,以道念为先,真实参学者乎?徒被邪师所惑,猥覆\note{歪曲遮挡}正解,空枉自狂,久沉迷乡,以何而长般若\note{智慧}正种,得得道之时哉?贫道今既以云游萍寄为事,将参访何山川\note{我在四处游历,求法者如何能寻得正法}?以怜是故,遂将于大宋国所闻之禅林风规\note{风范规仪},禀持之知识玄旨,记之集之,留与参学闲道\note{非正道}之人,令知晓佛家之正法,此乃真诀也。

曰:大师释尊\note{释迦牟尼佛},灵山会上付法迦叶\note{摩诃迦叶},祖祖正传而至菩提达摩\note{初祖}尊者。尊者亲赴神丹国\note{中国},付法慧可\note{二祖}大师,是东地佛法传来之始也。

如是单传\note{一祖传一祖},亲至六祖大鉴禅师\note{慧能或惠能}。是时,真实之佛法,方流演东汉,不拘节目之旨,于焉而显\note{形成}。于时六祖有二神足,为南岳怀让、青原行思。皆传持佛印\note{佛心如印契永久不变},同为人天导师。其二派流通,善开五门,谓法眼宗、曹洞宗、云门宗、临济宗\note{还有沩仰宗}也。现在大宋国,独临济宗遍于天下。五宗虽异,唯一佛心印也。

大宋国自后汉以来,虽教籍\note{佛教典籍}垂迹\note{佛、菩萨化身说法},广播天下,然雌雄未定。祖师西来之后,直截葛藤根源,纯一佛法,于焉弘通。亦愿我国有如是之事也。

谓住持佛法之诸祖并诸佛,皆以端坐、依行自受用三昧为其开悟之正道。西天\note{印度}东地\note{中国},得悟之人,概依其风\note{风气习惯}。此乃因师资\note{师徒}秘密正传妙术,禀持真实秘诀之故也。

宗门\note{禅宗门庭}正传\note{正宗传授}云:此单传正直之佛法,最上中之最上也。自参见知识\note{禅宗大师}始,勿须更烧香、礼拜、念佛、修忏、看经;只管打坐,得身心脱落\note{超脱烦恼}。

人虽\note{即使}一时\note{一时半刻}于三业\note{身、口、意三种活动}标\note{示现}佛心印,端坐于三昧时,则遍法界皆为佛印,尽虚空悉成证悟,是故,于诸佛如来,增本地之法乐,新觉道之庄严。及十方\note{东、西、南、北、东南、西南、东北、西北、上、下十个方位}法界\note{各种事物的现象及其本质}、三途\note{地狱、饿鬼、畜生}六道\note{“三途”加修罗、人、天}之群类,皆一时身心明净,证大解脱地\note{境界},现本来面目,诸法皆会证正觉,万物共使用佛身,倏然一超证会之边际,端坐觉树王\note{菩提树},一时转无等等大法轮\note{传法},开演究竟\note{最终、终极}之深般若。

此等等正觉\note{正确的觉悟},更回光返照\note{返回你身上},冥资通合\note{暗中资助}故,此坐禅人,便霍尔\note{疾速}身心脱落,截断从来\note{之前}杂秽\note{杂乱不纯}之知见\note{认知见解}思量\note{思虑考量},会\note{领会}证\note{证悟}天\note{天然}真\note{本真}佛法,悉于\note{在全部}遍\note{遍布}微尘际\note{极细小的地方}之诸佛如来道场,助发\note{帮忙发起}佛事\note{礼佛修道之事},广被\note{广泛提供}佛之向上机\note{礼佛的机会},激扬\note{推崇宣扬}佛之向上法\note{学佛的法门},是时,十方法界之土地、草木、墙壁、瓦砾,亦皆作佛事;蒙其所兴之风水利益\note{因以上佛事而受益}者,皆被甚妙不可思议之佛化\note{佛之化身}所冥资,显其亲证\note{展示佛所亲自证悟的};受用此水火之群类者,皆周旋本证之佛化\note{流转于本初证悟之佛的化身中}。所以,与此等群类共住同语者,亦悉皆互备\note{得到}无穷之佛德\note{功德},辗转\note{不断}广作\note{广泛修行},使无尽、无间断、不可思议、不可称量之佛法流通于遍法界之内外。然则,令诸等(坐禅)当\note{中}人不被知觉所昏\note{迷惑}者,以\note{因}于\note{在}静中无造作\note{执著攀缘},直证\note{直接证悟}故也。若如凡流\note{普通人}想\note{所认为的},以修\note{坐禅}证\note{证悟}为两段\note{两码事},则各互当觉知\note{则可以分别认知}。若混入觉知\note{如果可以分别认知},即非证则\note{证悟的本质},以于证则迷情不及故\note{证悟的本质非分别之心可以企及的}。
