% -*- coding: utf-8 -*-

\input macros

\beginchaptern{办道话}{修道篇}

\origpageno=1
\pageno=1

诸佛如来,皆单传妙法,证\word{阿耨菩提}{阿耨多罗三藐三菩提,无上正等正觉},有最上无为之妙术,是唯佛授之于佛而无\word{旁邪}{误入歧途},\word{即}{而}自\word{受用三昧}{自己受益的禅定},是其标准也。

游化\note{云游教化,传授}此三昧,当以\word{端}{肩平背直}坐参禅为正门。是法虽\word{人人分上丰备}{佛性人人生而具备},然未修者不\word{现}{显现出},未证者不\word{得}{体验到}。\word{放则满手}{不握着它反而充满双手},\word{岂一多之际乎}{无数量的限制}?\word{语则溢口}{言语无法尽述},纵横无限。

诸佛常\word{住持}{住留}此中,\word{各个方面不遗知觉}{不留痕迹};群生常\word{受用}{受益}此中,\word{各个知觉方面不露}{而不自知}。

今授之\word{功夫}{身体力行}\word{办道者}{修道方法,此处指坐禅},乃\word{于}{在}\word{证}{证悟}上现万法,于\word{出路}{解脱之道上}行\word{一如}{真如,不异之真理}也。其\word{超关脱落}{超越烦恼,解脱生死}时,岂拘此\word{节目}{木干交接处为节,纹理纠结处为目,此处指束缚人的概念理论}乎?

予自发心求法以来,于我朝诸方,寻访知识,因见\word{建仁}{京都建仁寺}\word{全公}{佛树房明全,荣西的弟子,宋时与道元一同入宋求法,后客死在中国},相随霜华忽历九载,\word{聊}{略}闻临济家风。全公者,祖师\word{西和尚}{明庵荣西,曾入宋求法,将临济宗传入日本,被奉为日本临济宗创始人}之上足,独正传无上之佛法,余辈不敢并比。

予又赴大宋国,访知识于\word{两浙}{浙江},闻家风于\word{五门}{禅宗五宗:曹洞宗、临济宗、沩仰宗、法眼宗、云门宗}。遂参见\word{太白峰}{浙江宁波天童山}\word{净}{如净}禅师,一生参学大事于兹了毕。尔后,大宋\word{绍定}{宋年号}初,返回本乡,即以弘法救生为怀,\word{尚如}{犹如}重\word{坦}{担}荷肩。

然放下\word{弘通}{弘扬}(佛法)之心,以待\word{激扬之时}{风云际会之时}故。且\word{云游萍寄}{四处游历},将\word{期}{希望}\word{闻}{传播}先哲之风范。但有\word{自}{自身}不拘名利,以道念为先,真实参学者乎?徒被邪师所惑,\word{猥覆}{歪曲遮挡}正解,空枉自狂,久沉迷乡,以何而长\word{般若}{智慧}正种,得得道之时哉?贫道今既以云游萍寄为事,将参访何山川\note{我在四处游历,求法者如何能寻得正法}?以怜是故,遂将于大宋国所闻之禅林\word{风规}{风范规仪},禀持之知识玄旨,记之集之,留与参学\word{闲道}{非正道}之人,令知晓佛家之正法,此乃真诀也。

曰:大师\word{释尊}{释迦牟尼佛},灵山会上付法\word{迦叶}{摩诃迦叶},祖祖正传而至\word{菩提达摩}{初祖}尊者。尊者亲赴\word{神丹国}{中国},付法\word{慧可}{二祖}大师,是东地佛法传来之始也。

如是\word{单传}{一祖传一祖},亲至六祖\word{大鉴禅师}{慧能或惠能}。是时,真实之佛法,方流演东汉,不拘节目之旨,于焉而\word{显}{形成}。于时六祖有二神足,为\uline{南岳怀让}、\uline{青原行思}。皆传持\word{佛印}{佛心如印契永久不变},同为人天导师。其二派流通,善开五门,谓法眼宗、曹洞宗、云门宗、临济宗\note{还有沩仰宗}也。现在大宋国,独临济宗遍于天下。五宗虽异,唯一佛心印也。

大宋国自\word{后汉}{东汉}以来,虽\word{教籍}{佛教典籍}\word{垂迹}{佛、菩萨化身说法事迹},广播天下,然雌雄未定。祖师西来之后,直截葛藤根源,纯一佛法,于焉弘通。亦愿我国有如是之事也。

谓住持佛法之诸祖并诸佛,皆以端坐、依行自受用三昧为其开悟之正道。\word{西天}{印度}\word{东地}{中国},得悟之人,概依其\word{风}{风气习惯}。此乃因\word{师资}{师徒}秘密正传妙术,禀持真实秘诀之故也。

\word{宗门}{禅宗门庭}\word{正传}{正宗传授}云:此单传正直之佛法,最上中之最上也。自参见\word{知识}{禅宗大师}始,勿须更烧香、礼拜、念佛、修忏、看经;只管打坐,得\word{身心脱落}{超脱烦恼}。

人\word{虽}{即使}\word{一时}{一时半刻}于\word{三业}{身、口、意三种活动}\word{标}{示现}佛心印,端坐于三昧时,则遍法界皆为佛印,尽虚空悉成证悟,是故,于诸佛如来,增本地之法乐,新觉道之庄严。及\word{十方}{东、西、南、北、东南、西南、东北、西北、上、下十个方位}\word{法界}{各种事物的现象及其本质}、\word{三途}{地狱、饿鬼、畜生}\word{六道}{“三途”加修罗、人、天}之群类,皆一时身心明净,证大解脱\word{地}{境界},现本来面目,诸法皆会证正觉,万物共使用佛身,倏然一超证会之边际,端坐\word{觉树王}{菩提树},一时转无等等大法轮\note{传法},开演\word{究竟}{最终、终极}之深般若。

此等\word{等正觉}{正确的觉悟},更回光返照\note{返回你身上},\word{冥资通合}{暗中资助}故,此坐禅人,便\word{霍尔}{疾速}身心脱落,截断\word{从来}{之前}\word{杂秽}{杂乱不纯}之\word{知见}{认知见解}\word{思量}{思虑考量},\word{会}{领会}\word{证}{证悟}\word{天}{天然}\word{真}{本真}佛法,\word{悉于}{在全部}\word{遍}{遍布}\word{微尘际}{极细小的地方}之诸佛如来道场,\word{助发}{帮忙发起}\word{佛事}{礼佛修道之事},\word{广被}{广泛提供}佛之向上机\note{礼佛的机会},\word{激扬}{推崇宣扬}佛之向上法\note{学佛的法门},是时,十方法界之土地、草木、墙壁、瓦砾,亦皆作佛事;蒙其所兴之风水利益\note{因以上佛事而受益}者,皆被甚妙不可思议之\word{佛化}{佛之化身}所冥资,\word{显其亲证}{展示佛之所亲自证悟的妙谛};受用此水火之群类者,皆\word{周旋本证之佛化}{流转于本初证悟之佛的化身中}。所以,与此等群类共住同语者,亦悉皆互\word{备}{得到}无穷之\word{佛德}{佛之功德},\word{辗转}{不断}\word{广作}{广泛修行},使无尽、无间断、不可思议、不可称量之佛法流通于遍法界之内外。然则,令诸等(坐禅)\word{当}{中}人不被知觉所\word{昏}{迷惑}者,\word{以}{因}\word{于}{在}静中无\word{造作}{执著攀缘},\word{直证}{直接证悟}故也。若如\word{凡流}{普通人}\word{想}{所认为的},以\word{修}{坐禅}\word{证}{证悟}为\word{两段}{两码事},\word{则各互当觉知}{则可以分别认知}。\word{若混入觉知}{如果可以分别认知},即非\word{证则}{证悟的本质},\word{以于证则迷情不及故}{因为证悟的本质非分别之心可以企及的}。

又于心境虽有静中证入、悟出,而以自受用之境界故,不动一尘,不坏一相,广大之佛事,作甚深微妙之佛化。此化道所及之草木土地,皆放大光明,说深妙法,无有穷尽。草木墙壁能为凡圣含灵宣扬,凡圣含灵返为草木墙壁演畅。自觉觉他之境界,自本以来,具证相,无有欠缺;证则行之,而无懈怠之时。

由是,虽一人一时坐禅,以其与诸法相冥资,与诸时互圆通故,于无尽法界中,于过去、未来、现在,作常恒之佛化道事。彼彼共为一等之同修、同证也,非唯坐上之修,如击空之响,撞之前后,妙声绵绵。何限如此乎?百头(万事)皆于本来面目具本来修行,非可臆测。

须知即便十方无量恒河沙数诸佛,皆共励力,以佛之智慧计知一人坐禅之功德,亦不得其边际也。

今闻此坐禅之功德高大,(然)愚人疑谓:佛法有多门,何故偏劝坐禅耶?

示曰:是以佛法之正门故。

问曰:何故独为正门?

示曰:大师释尊,诚正传得道之妙术;又,如来三世,皆由坐禅而得道,故即相传之正门也。不仅如此,西天东地之诸佛,皆由坐禅而得道也。是故今以正门示人天。

问曰:(坐禅)或乃如来之正传妙术,又因寻祖师之踪迹,诚非凡虑所及。然读经、念佛者,亦应自为证悟之因缘;唯空坐而无所为,依何而得悟耶?

示曰:汝今若将诸佛之三昧、无上之大法认作空坐而无所为,乃谤大乘人也。迷如井深,如居大海而言无水。既面壁而坐,即是安坐于诸佛自受用三昧也,此非广大之功德乎?可哀眼未开,心犹在醉!

大凡诸佛境界,不可思议也,非心识所能及,况不信劣智者之所知哉?唯正信之大机能得入也。不信之人,虽诲之,亦难受持。灵山尚有退亦佳矣之辈。大凡心中起正信,即可修行、参学。不然则须暂且休止,当恨从昔以来无佛法泽被。

又,读经、念佛等修行所得之功德,汝知么?只动舌头,喊声音,以为佛事功德者,实可悲矣!以是拟为佛法者,(佛法)愈远。又翻阅经书,则明佛教诲顿渐修行之仪则,依教修行,必得证悟。非徒费思量念度,拟得菩提之功德也。愚痴而造千万诵口业,以为至佛道者,如北辙向越国也,又同圆孔容方木也。看文而暗修道者,如见医方而忘调药也,何益之有?口声喋喋不休,如春田青蛙,昼夜鸣叫,终亦无益矣!况深惑名利之徒,难舍此等之事。其乃贪利之心甚深故。昔既有之,今世无乎?最可怜矣!

只当知七佛之妙法,若于得道明心之宗匠、契心证会之学人相从正传,则的旨现前,可禀持也。非学文字之法师可知及也。是故,当息止此疑迷,依正师之教,坐禅办道,得证诸佛自受用三昧。

问曰:今我朝所传之法华宗、华严教,共为大乘之究竟。况如真言宗,乃为毗卢遮那如来亲传金刚萨埵,师资不乱。其所谈之旨,谓即心即佛,是心作佛,不待多劫修行,于一坐之中,得五佛之正觉,堪称佛法中之极妙也。然今所言之(坐禅)修行,有何殊胜?(何故)舍彼等诸宗,而独劝此哉?

示曰:当知佛家中,无论教之殊劣,不抉法之浅深,但知修行之真伪。有引于花草山水而入佛道者,有执土石砂砾而禀持佛印者。况乎广大文字,溢满万象而尚盛余,转大法轮而又摄于一尘。然则即心即佛之言,乃水中之月;即坐成佛之旨,更亦镜中之影,不得拘泥于言语之巧。今所劝勉直证菩提之修行,示佛祖单传之妙道,以令其成真实之道人也。

又,传授佛法者,当以契证之人为其宗师。数文字之学者,不足为其导师,如一盲引众盲也。今此佛祖正传之门下,皆敬仰得道契证之哲匠,令佛法住持。是故,冥阳神道亦来皈依,证果之罗汉亦来问法,无不授以开明各个心地之手。余门未所有闻也。唯佛弟子可学佛法。

又当知吾等虽从本以来,于无上菩提,无欠无缺,恒常受用,但因不得承当,故乱习起知见之事,又因以其为(实)物,故大道白蹉跎。依此知见,空花纷然,或执十二轮转,或念二十五有之境界,或执三乘五乘、有佛无佛之见,无有尽事。不得习如是知见以为佛法修行之正道也。然今正依佛印,放下万事,一向坐禅时,则横超迷悟、情量之边际,不关凡圣之道,悉逍遥于格外,受用大菩提也。彼执文字筌蒂者,岂能比肩耶?

问曰:三学之中有定学,六度之内有禅定,皆是一切菩萨初发心时所研习,不分根之利钝,悉皆修行。今之坐禅,亦当为其之一,依何而言(坐禅)之中集有如来正法耶?

示曰:因今以如来一大事之正法眼藏、无上大法谓为禅宗,故有此问出。

当知,此禅宗之号,起于神丹以东,而竺乾尚不见闻。初达摩大师于嵩山少林寺九年面壁间,道俗尚不知佛道正法,名以坐禅为宗之为婆罗门。后代代之诸祖,皆亦常专于坐禅。见之愚昧俗家,不知其实,概言其为坐禅宗。今之世上简坐语,单言禅宗。其意明见于诸佛之广语。不可与六度及三学之禅定并称。

此佛法相传之嫡意,无一代隐之。如来昔在灵山会上,以正法眼藏、涅槃妙心、无上大法独付迦叶尊者之仪式,现今天上界之天人大众,耳闻目睹者尚在,不应存疑。大凡佛法,由彼天人大众永久护持,其功迄今不废。是故,当知此(坐禅)乃佛法之全道也,无物可与其并语。

问曰:佛家依何于四仪中独尊坐,劝禅定而言入证哉?

示曰:昔来之诸佛,相继修行,证入之道,极难知尽。若寻其故,当知唯以佛家所用者为其故也,不须向此外寻之。但祖师叹曰:“坐禅乃安乐法门也。”可知岂四威仪中之安乐故乎?况乎非一佛二佛修行之道也,诸佛诸祖皆有此道也。

问曰:此坐禅行,未证会佛法者,可坐禅办道摄其取证;既明佛正法者,于坐禅有何所待?

示曰:虽言痴人面前不得说梦,山子手里难与舟柁,然更须示训。

夫谓修证非一者,即外道之见也。佛法之中,修证是一等也。即今亦是证上之修故,初心之办道即是本证之全体。是故教授修行之用心,谓于修之外不得更待有证,以是直指之本证故也。既修是证,证无际限;已是证而修,修无起始。故此,释迦如来、迦叶尊者,皆受用于证上之修;达摩大师、大鉴高祖,同于证上之修所引转。住持佛法者,悉皆如是。

既有不离证之修,吾等幸单传一份之妙修,初心办道,即得一分之本证于无为之地。当知令不离证之修不得污染,祖师常诲不得怠慢修行。放下妙修,则本证溢满手中;出离本证,则妙修行于通身。

又,(予)在大宋亲见诸方禅院皆构有坐禅堂,安置五百六百及一二千僧,令其昼夜坐禅。当(予)向其席主、传佛心印之宗师问佛法之大意,则教以修行非两段之旨。

是故,不仅门下之参学,且求法之高流,向佛法中求真实之人,不管初心后心,不论凡人圣人,劝其当依佛祖之教,循宗匠之道,坐禅办道。

岂不闻乎?祖师云:“修证即不无,污染即不得。”又云:“见道者修道。”当知于得道中可修行。

问曰:我朝先代弘教之诸师,其皆入唐传法时,何故舍比旨而独传教耶?

示曰:昔之人师未传此法者,以时节未至之故也。

问曰:彼上代之师会得此法乎?

示曰:会则通焉!

问曰:或言:“莫叹生死,出离生死,有疾速之道,谓知心性常住之理也。其旨谓此身体者既有生,必被灭所迁,然此心性则无灭事。若能知我身中有不被生灭所迁之心性,是本来性故,此身即是假相也,死此生彼无定。心即常住,过去、未来、现在,无有变异。如是知者,称脱离生死。凡知此旨者,从来之生生死死永绝,离此身时,即入性海。潮流于性海时,如诸佛如来,具玄妙德相。今虽知之,因前世之妄业所造之身体故,不与诸圣相等。未知此旨者,当久回生死。是故,当速了知心性常住之旨。而只管闲坐,了却一生,有何可待?”如是之言,诚契诸佛诸祖之道乎?如何?

示曰:今所言之见解,全非佛法,先尼外道之见也。

曰:彼外道之见,谓于己身中有一灵知,其灵知遇缘则分辨好恶,分别是非。知痒痛,知苦乐者,皆彼灵知之力也。然彼灵性,此身坏时,即移住他处,是故见之虽此处已来,然实则他处有生,永不坏灭而常住。彼外道之见,如是也。

然则,习此见而为佛法者,比之持瓦砾而为金宝者更愚。迷痴可悲,无物可喻。大唐国之慧忠国师尚有深训。今计心常相来之邪见,以为与诸佛之妙法等同,作如是之生死本因,以为脱离生死者,岂不愚乎?最可悲矣!但知是外道之邪见,不得触耳。

事不得已,今尚垂怜悯,以救汝等之邪见。须知佛法从本以来,谈身心一如,性相不二。此乃西天东地所共知,不可疑之。况乎谈常住门时,万法皆常住,不分别身心;谈寂灭门时,诸法皆寂灭,不分别性相。然则,何故言身灭心常?不违正理乎?非但如此,且须觉了生死即涅槃也,不于生死之外谈涅槃也。况虽以此领解心离身而常住,妄计为已离生死之佛智者,而此领解知觉之心乃生灭,全非常住,此不虚妄乎?

尝观身心一如之旨,乃佛法之所常谈也。然则,何故言此身灭时,心独离身而不灭耶?若有一如时,有非一如时,则佛说自成虚妄也。又,以为生死当为离舍之法者,则成谤佛法罪,焉不慎哉?

当知佛法云心性大总相法门者,含摄一大法界而不分别性相,不谓生灭。及至菩提涅槃,无非心性。一切诸法,万象森罗,皆但是一心而无不含藏,无不兼带。此诸法门,皆平等一心也。谈之无异违者,即是知佛家心性之样子也。

既如是,于此一法岂能分别身心,分别生死涅槃哉?既为佛子,勿须耳听狂人喋谈外道邪见。

问曰:专务坐禅之人,必严净戒律乎?

示曰:持戒梵行,乃禅门之规矩也,佛家之家风也。未受戒,或破戒者,不无其分。

问曰:修此坐禅人,更兼修真言、止观之行,可有无妨碍耶?

示曰:(予)在唐时,因向宗师问真诀,云:西天东地之古今,正传佛印之诸祖,皆未闻兼修如斯之行。诚哉!不专一事,则不达一智。

问曰:此(坐禅)之行,在俗之男女亦可修耶?独出家人修耶?

示曰:闻祖师言:“领会佛法者,不问男女贵贱。”

问曰:出家人诸缘速离,不障坐禅办道;在俗者,繁于世务,如何一向修行而契无为之佛道耶?

示曰:大凡佛祖哀怜之余,开广大慈悲门,为令一切众生而得证入,人天谁不得证入?是故追寻古今,其证乃多。且代宗、顺宗在帝位,日理万机,坐禅办道而会佛祖之大道。李相国、防相国,同为辅佐之臣位,乃为一天之股肘,坐禅办道而证入佛祖大道。此唯由心志之有无,不关身之在家出家。又,深究事之殊劣之人,自有所信焉。况乎以为世务障佛法者,只知世(法)中无佛法,不知佛法中无世法也。

近于大宋,有谓冯相公者,擅长祖道,大官也。后作诗,自述曰:
\begintt
    公事之余喜坐禅,少曾将胁到床眠;虽然现出宰官相,
因以深志于佛道。
\endtt
\noindent 此人虽是官务繁忙之身,因深志于佛道,故能得道也。当以他鉴已,以古鉴今。

大宋国,现今之国王大臣、士俗男女,无不寄心于祖道者。武门文家,亦皆志向参禅学道。有志者,多必开明心地。世务不碍佛法,由是自知。

真实之佛法若弘通国家,诸佛诸天常作卫护故,王化太平也。圣化太平,佛法则得其力也。

又,释尊在世时,有逆人、邪见得道者;祖师会下,有葛者、樵翁开悟者。况乎其他人哉?唯当寻正师之教道矣!

问曰:此(坐禅)行,于今末代恶世修之亦可得证乎?

示曰:教家以名相为事,尚于大乘实教,不判正像末法,谓修之悉皆得道。况此单传正法,不论初学后学,同受用自家财珍也。证之得否,乃修者自知,如用水之人,冷暖自知。

问曰:或言:“佛法之中,若了达即心是佛之旨,口不诵经典,身不行佛道,于佛法亦无甚欠缺,但知佛法从本以来在之自己,以此为得道圆满,勿须更向他人索求,况乎烦于坐禅办道哉?”

示曰:此言最不足取。若如汝言,有心者,由谁教此旨,有不知者。

当知,佛法诚乃息止自他之见而学也。若以知自己即佛为得道,则释尊昔日不曾烦于化道也。且当以古德之妙则证之。

昔则公监院,居法眼祖师会下,法眼禅师问曰:“则监寺,汝在我会下几多时?”则公曰:“吾侍师会下,已三年。”禅师曰:“汝是后生,何不常向吾问佛法?”则公曰:“某甲不敢欺和尚,曾在青峰禅师处时,于佛法已了达安乐处。”禅师曰:“汝以何言而得入?”则公曰:“某甲曾问青峰‘如何是学人底自己?’青峰曰:‘丙丁童子来求火。’”法眼云:“好言也!但恐汝不会。”则公曰:“丙丁属火,以火更求火,会似以自己求自己也。”禅师云:“当知汝不会也。佛法若如是,则不传至今日也。”

于是,则公燥闷,即起。至路途中,念禅师乃天下之善知识,又是五百人之大导师,谏我非,定有长处。乃归禅师下,忏悔礼谢,问曰:“如何是学人底自己?”禅师云:“丙丁童子来求火。”则公于言下大悟佛法。

当知以自己即佛之领解谓知佛法者,非也。若以自己即佛之领解为佛法,禅师则不以前言引导之,亦不当如是诫之。唯自见善知识始,当咨问修行仪则,一向坐禅办道,勿须心留一知半解。佛法妙术,其不虚妄。

问曰:闻乾唐之古今,有闻竹声而悟道者,有见花色而明心者,况乎释迦大师,见明星而证道;阿难尊者,因刹竿倒而明法。乃至六代后,五家之间,于一言半句下而明心者亦众。彼等未必皆只坐禅办道也。

示曰:当知古今见色明心,闻声悟道之人,皆于办道不拟议思量,直下无第二人也。

问曰:西天及神丹国,人本质直,因中华所至;教化佛法,皆能疾速证会人道。我朝则自昔以来,人之少仁智,正种难积,蕃夷所至,何不憾哉!又,此国之出家人,亦劣于大国之在家人,举世皆愚,心量狭小也,深执有为之功德,好事相之善,如是之辈,即便坐禅,亦可直证得佛法乎?

示曰:诚如所言,我国之人,仁智未广,人又迂曲,即便示正直之法,甘露反当毒药。名利易趋,惑执难除。虽如是,证入佛法者,不必皆以人天之世智作出世之舟航。佛在世时,有因皮球打而证四果者,有着袈裟而明大道者,皆为愚暗之辈,痴狂之畜类也。唯以正信之所助,即有离迷之道。又有见痴老比丘默坐而设斋之信女开悟者。此非依智,非依文,非待语,唯得助于正信也。

又,释教广被三千界,恰二千年前后,刹土种种,未必皆为仁智之国,人亦未独利智聪明。然则,如来之正法,从本以来具有不可思议之大功德力,时节若至,则弘通其刹土。人若诚然正信修行,不分利钝,悉等得道也。勿须以为我朝非仁智之国,人之知解至愚,而不可会解佛法也。况人皆丰备般若正种,但以承当者稀,受用者寡尔。

前之问答往来,宾主相交,不胜杂然,多有无花之空而使花开之嫌。然则,此国于坐禅办道,尚未传其宗旨,于有志者,当可悲矣!是故,聊集异域之见闻,记留明师之真诀,以教有参学之愿者。此外,丛林之规范及寺院之格式,今无示说之暇,亦不尽欲言。

大凡我朝,处龙海以东,云烟虽遥,然自钦明、用明之前后,秋方佛法东渐,此乃人之所庆幸也。然则,名相、事缘繁杂,乱于修行。今以被衣缀盂为生涯,结茅庵于青岩白石山奥,端坐修炼,直现佛向上事,一生参学大事,当即究竟。此即龙牙之赦诫,鸡足之遗风也。其坐禅仪则,当依前嘉禄年间撰集之《普劝坐禅仪》。

夫佛法弘通国中,虽当待王赦,然复思灵山遗嘱,现今出现于百万亿刹之王公、相将,悉皆蒙受佛赦,终身不忘护持佛法之素怀而生来者也。其统领之域,无不为佛之国土。是故,为令佛祖之道流通,不必择地待缘,只念今始是也。

是故,记集此文,留待志愿佛法之哲匠,并云游萍寄之参学道流。

\hskip2em 宽喜辛卯中秋日

\hfill{入宋传法沙门道元记\hskip4em}

\endchapter